%% quantum's style

\documentclass[a4paper,twocolumn,11pt]{quantumarticle}

\pdfoutput=1
\usepackage[utf8]{inputenc}
\usepackage[english]{babel}
\usepackage[T1]{fontenc}
\usepackage{amsmath}
\usepackage{hyperref}
%
%\usepackage{tikz}
%\usepackage{lipsum}


%% our previous style and packages

%\documentclass[two column]{article}
%\usepackage[utf8]{inputenc}
\usepackage{graphicx}
\usepackage[dvipsnames]{xcolor}
%\usepackage[colorlinks=true,linkcolor=Blue,hypertexnames=false]{hyperref}
\usepackage{braket}
\usepackage{bbold}



\usepackage[numbers,sort&compress]{natbib}
%\usepackage[style=phys]{biblatex}
\usepackage{amssymb}
%\usepackage{amsmath}
\usepackage{placeins}
\usepackage{subcaption}

\usepackage[left=23mm,right=23mm,top=35mm,columnsep=15pt]{geometry}

\usepackage{pgfplotstable}
\usepackage{array}

\usepackage[qm]{qcircuit}

\newcommand{\caro}[1]{\textcolor{red}{[#1]}}
\newcommand{\jovan}[1]{\textcolor{blue}{[#1]}}
\newcommand{\steve}[1]{\textcolor{purple}{[#1]}}




\title{Notes on $\mathbb{Z}_2$-SPT preparation protocols}
\author{Jovan Jovanovi\'c}
\affiliation{Rudolf Peierls Centre for Theoretical Physics, Parks Road, Oxford, OX1 3PU, UK}
\orcid{0000-0002-2508-3207}
\author{Carolin Wille}
\affiliation{Rudolf Peierls Centre for Theoretical Physics, Parks Road, Oxford, OX1 3PU, UK}
\orcid{0000-0002-9764-6937}
\author{Steven H. Simon}
\affiliation{Rudolf Peierls Centre for Theoretical Physics, Parks Road, Oxford, OX1 3PU, UK}
\orcid{0000-0001-7757-5978}



\date{11.08.2023.}

\begin{document}

\maketitle
\begin{abstract}
TBC  
\end{abstract}
\tableofcontents



\section{Overview}

Symmetry protected topological phases (SPTs) can be characterised by the properties of their fixed point wave-functions. Those properties are: \begin{enumerate}
\item Symmetric under an onsite representation of some symmetry group $G$.
\item It's parent Hamiltonian is gapped and it's ground state is unique, if the system lives on a closed manifold.
\item  If the manifold has an edge, a suitable parent Hamiltonian either supports a gapless edge mode or symmetry is spontaneously broken at the edge.
\end{enumerate}

Given the onsite symmetry group $G$ in $2+1$ dimensions we can classify all bosonic SPTs by the third cohomology group $\mathcal{H}^{3}(G, U(1))$ \cite{spt_coho_org}. Bosonic in this sense means that the elementary degrees of freedom commute, i.e. spin lattices. 

Given an element of the cohomology group $[\omega] \in \mathcal{H}^{3}(G, U(1))$ we can construct the fixed point wavefunction on a closed manifold.

For the case of concreteness we will immediately limit ourselves to the case at hand, $G = \mathbb{Z}_2$, its third cohomology group is $\mathcal{H}^{3}(\mathbb{Z}_2, U(1)) = \mathbb{Z}_2$. The two elements of the cohomology group represent the trivial paramagnet $\ket{\Psi_0}$ and our target state (the nontrivial SPT) $\ket{\Psi_1}$.

The two states are deceptively similar, given a spin-$1/2$ configuration on any lattice of $N$ sites the two states are: \begin{gather}
\ket{\Psi_0} = \sum_{\{s_i\}\in\{0,1\}^N}\ket{\{s_i\}}, \\ \ket{\Psi_1} = \sum_{\{s_i\}\in\{0,1\}^N}(-1)^{N_{d.w.}(\{s_i\})}\ket{\{s_i\}},
\end{gather} where $N_{d.w.}(\{s_i\})$ is the number of domain walls in a given configuration $\{s_i\}$. It is in the definition of this quantity that the embedding manifold plays a role.

\subsection{The fixed point Hamiltonian}

\section{State preparation on manifolds without boundary}

We will introduce our lattice. 




%\printbibliography

\bibliographystyle{quantum}
\bibliography{bibliography}

\end{document}
